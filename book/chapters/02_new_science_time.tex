\chapter{Время новой науки}

XX век был поистине богатым научными открытиями в самых разных областях науки. Ученые как никогда приблизились к пониманию механики как микро, так и макро процессов окружающего мира. В биологии был обнаружен и описан основной строительный блок всего живого - молекула ДНК. Стремительно начала развиваться генная инженерия, находя приложения в самых разных отраслях человеческой деятельности. В физике были открыты общая и специальная теории относительности, квантовая механика. Выдающиеся достижения физиков и биологов активно освещались в прессе и практически сразу становились предметом жаркого обсуждения даже людьми, далекими от мира науки и в лучшем случае довольно приблизительно понимающими, о чем идет речь. 
Подобного, к сожалению, нельзя сказать об отношении к достижениям математики XX века - кроме самих математиков и, пожалуй, некоторых физиков, о них не знал практически никто. А они были поистине впечатляющими, вполне сравнимыми по потенциальной мощи с квантовой механикой или открытием ДНК. Стоит упомянуть хотя бы появление и активное использование компьютеров, необходимых для сложных расчетов тогда и распространенных повсеместно сейчас.

Отсутствие должного освещения открытий математики отчасти связано с самой спецификой данной науки. Лишь в редких случаях по-настоящему сложную математическую теорию можно объяснить широкому кругу людей-непрофессионалов. Чувство красоты математических рассуждений, доказательств и окончательных выводов необходимо упорно воспитывать в себе некоторое время, прежде чем появится понимание того, что стоит за длинными формулами и придет осознание того, как полученные выводы можно применить на практике.
Данная книга призвана восполнить этот пробел и рассказать, какую роль сыграли математики в знаменитом манхэттенском проекте, явившим миру всю мощь ядерной энергии. Я попытаюсь осветить мат. аппарат, который использовался при расчетах, связанных с конструированием атомной и водородных бомб, уделяя особое внимание методам, созданным именно в процессе работы над проектом “Манхэттен”.

Книга рассчитана на широкий круг читателей, интересующихся математикой и ее приложениями в ядерной физике. Книга будет интересна студентам и аспирантам физико-математических специальностей, а также просто интересующиеся тематикой атомной физики начала-середины XX века и применяемого там мат. аппарата. От читателей в большинстве случаем требуются лишь общие знания об основных понятиях математики - множествах и отображениях. Для понимания наиболее сложных моментов книги будет полезна специальная подготовка в рамках не ниже 2 курса физико-математических специальностей, общие знания по математическому анализу, теории вероятностей, дифференциальным уравнениям и функциональному анализу.






Формулу $E = mc^2$ как мантру может повторить практически любой современный человек.
Многие из нас так или иначе слышали о ней еще в детстве, не подозревая, что же она в действительности означает.
....


Попытки разобраться в сути какого-либо уже исследованном кем-то ранее явлении реального мира чем-то напоминают процесс очистки гипотетического фрукта с многослойной кожурой.
Первым и самым простым слоем являются личный опыт, мнения других людей и ``авторитетных'' источников о данном вопросе. 
На этом, собственно, можно и остановиться, сказав, что достаточно разобрались в вопросе.

Если полученные ответы нас не устраивают, не понятны, либо не полны и желание разобраться в сути явления не угасло, то придется перейти к следующему слою - предметной области явления, например, физике.
Необходимо хотя бы в общих чертах понять, что же именно происходит в интересующем нас явлении природы. 
Какие объекты в нем участвуют и по каким правилам взаимодействуют друг с другом. Какие моменты существенны, а какими можно пренебречь.
Продвинувшись в понимании физической сути процесса, мы 

Наконец, последний и традиционно самый трудный слой - математика явления.
Каждое явление природы имеет свой язык описания  ...  сложно .. вместо объектов - абстракции, вместо простых правил взаимодействия - сложные уравнения.



------------------------ IDEAS ------------------------ 

атомы - интуиция еще со времен древних греков, но дальше - перерыв почти на x000 лет связанный с тем, что увидеть объект своих измышлений уже не возможно.
прорыв - с появлением соответсвующих средств измерений, но тут ученых ждал очень большой сбрприз
до этого схема научных открытий в большинстве своем состояла в следующем - смотрели, измеряли, придумывали теорию основнную на уже известных аналогиях, потом совершенствовали приборы, и снова смотрели и придумывали анлогию и т.п.
в атомной физике известных аналогий не нашлось. Наблюдения зачастую в корне противоречили известным фактам о макромире. Любая попытка смотреть на макро-аналогии заканчивалась появлением множества противоречий теории с кспериментом и в конце концов полным провалом 


физика - ранее умение делать открытия зависело от того, насколько наблюдателен был ученый, насколько хорошо он умел проводить параллели между уже известными ялениями и только изучаемыми.
Движения огромных небесных тел описывалось исходя из аналогичных движений, которые можно было повторять в удобном масштабе в своей алборатории и т.п. [еще примеры]
Новая физика потребовала от ученых вообразить нечто не имевшее аналогов с ранее изученным в принципе. 
Это восхищало даже далеких от физики современников.
В математике такие штуки привыкли проворачивать довольно давно. 
Стефан Банах, один из создателей современной математики в ее [современном] виде, говорил "Хорошие математики видят аналогии, лучшие могут видеть аналигии между аналогиями". Сам он, безусловно, был одним из лучших.

 
слова "теория относительноси", "квантовая механик" носились в воздухе. Их можно было слышать  понимающих и истолковыва


