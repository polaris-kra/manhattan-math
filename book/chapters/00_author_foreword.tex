\chapter*{Предисловие автора}

6 и 9 августа 1945 года на Хиросиму и Нагасаки были сброшены атомные бомбы общей мощностью более 40 килотонн в тротиловом эквиваленте.
Фактически, в этом уже не было необходимости. 
Вторая Мировая война уже подходила к своему завершению. 
Германия капитулировала, Япония была как никогда близка к капитуляции.
Атомные бомбардировки были призваны в основном продемонстрировать всему миру беспрецедентную военную мощь США. 
Мощь, родившуюся как результат, пожалуй, самого амбициозного проекта в истории науки, объединившего сотни лучших ученых своего времени, - проекта ``Манхэттен''.

Применение атомного оружия не оставило равнодушным никого.
Весь мир активно обсуждал нравственные и политические вопросы использования атомной энергии в военных целях.
Многие из тех, кто призывал к созданию атомного оружия и в конце концов осознал, что Германия не обладает ничем подобным, стали активно пропагандировать не использовать его и даже прекратить исследования в этой области. 
Сам Альберт Эйнштейн, подписавший в 1939 году знаменитое "письмо Эйнштейна Рузвельту" с призывом к созданию атомного оружия, позднее отчаянно критиковал его разработку и применение в военных целях.

------------------------ IDEAS ------------------------ 

Однако, отвлечемся от моральных, нравственных, политических и других вопростов гуманитарного характера, неразрывно связанных с созданием и использованием атомной и водородной бомб. 






Однако интересна и другая, чисто ..... сторона этих событий. 

Именно благодаря беспрецедентной эффективности ядерного оружия репутация ученых среди простых обывателей взлетела до небес. 


Кто-то искренне поражался силе научной мысли, кто-то решительно критиковал столь безответственное, по его мнению, применение своих знаний.

До Манхэттенского проекта ученые воспринимались обычными людьми как чудаковатые, но безобидные и погрязшие в своих никому больше непонятных задачах. 
Образ эксцентричного ученого имел свой шарм, и то и дело всплывал то в фантастической литературе, то в кино.
После Второй Мировой войны отношение к ученым изменилось кардинально.
Весь мир узнал, насколько разрушительными могут быть идеи, рождающиеся на досках в университетах.


Многие из нас знали еще с детства, что е равно эм цэ в квадрате, не подозревая, что это в действительности означает.

Первый слой - физики

Второй - математики


