\chapter*{Предисловие автора}
\addcontentsline{toc}{chapter}{Предисловие автора}

6 и 9 августа 1945 года на Хиросиму и Нагасаки были сброшены атомные бомбы общей мощностью около 35 килотонн в тротиловом эквиваленте.
Взрывы произошли в 500 метрах над землей, сформировав сильнейшую ударную волну и выделив огромное количество тепла и радиации.
Более 180 тысяч человек погибло в момент взрыва и в последующие дни от ранений и лучевой болезни.
Через неделю Япония объявила о своей капитуляции, 2 сентября был подписан формальный акт.

Фактически, в атомных бомбардировках уже не было необходимости. 
Вторая Мировая война подходила к своему завершению. 
Германия капитулировала, Япония была как никогда близка к этому.
Атомные бомбардировки в основном были призваны продемонстрировать всему миру огромную военную мощь США, полученную вместе с новым оружием.
Мощь, явившуюся результатом, пожалуй, самого амбициозного проекта в истории науки, объединившего десятки лучших физиков и математиков своего времени - проекта ``Манхэттен''.

Результаты применения атомного оружия не оставили равнодушным никого.
Весь мир начиная с простых людей и заканчивая учеными и политиками активно обсуждал нравственные и политические вопросы использования атомной энергии.
Кто-то искренне поражался силе научной мысли, освободившей энергию атома.
Другие решительно критиковали столь безответственное, по их мнению, применение полученных знаний о природе.
Руководители стран Европы, Советского Союза и Востока понимали, что США впервые в истории удалось получить в свои руки оружие беспрецедентной разрушительной силы. 
Это был именно тот эффект, на который надеялись основные заказчики атомного проекта - политики и военные.  
Началась ядерная гонка, продлившаяся несколько десятилетий и не оконченная полностью по сей день.

Отношение самих ученых к успеху атомного проекта США было неоднозначным.
Многие из тех, кто призывал к созданию атомного оружия для победы над фашистской Германией и в конце концов осознал, что Германия не обладает ничем подобным, стали активно пропагандировать не использовать его и даже прекратить исследования в этой области. 
Сам Альберт Эйнштейн, подписавший в 1939 году знаменитое ``письмо Эйнштейна Рузвельту'' с призывом к созданию атомного оружия, позднее отчаянно критиковал его разработку и применение в военных целях.

Другие же ученые, напротив, продолжили активно работать в направлении увеличения эффективности и разрушительной мощи ядерного оружия.
Кто-то не видел своей личной вины за произошедшее в Хиросиме и Нагасаки, другие подозревали Советский Союз в активизации работ к этом направлении.
Иные понимали, что второго шанса работать бок о бок с лучшими умами своего времени над столь масштабными проектами им, возможно, уже не представится. 
Так или иначе, после событий августа 1945 года исследовательская лаборатория в Лос-Аламосе, в которой всего два года тому назад стартовал проект ``Манхэттен'', после небольшой передышки начала работу над еще более смертоносным оружием - водородной бомбой.
Далеко не все подразделения трудились в прежнем составе, но и того, что осталось, вполне хватило для успешного завершения и этого проекта. 
Полномасштабные испытания 1 ноября 1952 года явили миру мощь термоядерного оружия, гораздо более разрушительного, нежели атомное.
Испытательный взрыв, осуществленный США на атолле Эниветок, по меньшей мере в 1000 раз превосходил по мощности бомбу, сброшенную на Хиросиму.

Годом позже, в 1953-м, к термоядерной гонке присоединился и СССР, взорвав сначала более слабый заряд, чем у США, но быстро нарастив темпы. 
Затем в гонку включились Великобритания, Китай и Франция.
Мысли о повороте разрушительной мощи атома в мирное русло и разоружении возникли у политиков только во время Карибского кризиса в 1962 году, когда мир стоял на грани ядерной катастрофы. 
И лишь еще через десять лет, в 1972-м году, между СССР и США был подписан первый двусторонний договор, запрещающий сторонам наращивать ядерный арсенал.

Был и другой интересный эффект событий 40-х - 50-х годов XX века.
До Второй Мировой войны на ученых смотрели как на людей чудаковатых, но безобидных и погрязших в своих никому больше непонятных задачах. 
Образ эксцентричного ученого имел свой шарм, то и дело возникая в фантастической литературе и кино, но не более того.
Как позже писал участник Манхэттенского проекта, выдающийся физик XX века Ричард Фейнман о непопулярности науки среди простых людей: ``искусство мы уважаем больше, чем науку''.
Фейнман любил играть на барабанах бонго в местном баре и удивлялся тому, что ведущий представляет его как артиста, и никогда как атомного физика. 

После Второй Мировой войны отношение к ученым среди простых людей сильно изменилось.
Весь мир узнал, насколько разрушительными могут быть идеи, рождающиеся на кончике пера и университетских досках.
Человечество, на протяжении всей своей истории принимавшее дары науки в лучшем случае с неосознанной благодарностью, просто не было готово к столь кардинальным и потенциально опасным изменениям в их жизни.
У многих возник естественный страх перед той разрушительной мощью, которую наука без особого труда могла призвать в этот мир. 

О самом Манхэттенском проекте сказано и написано довольно много.
За более чем полвека с лица проекта исчезли почти все темные пятна с гифами секретности, стала понятна мотивация заинтересованных сторон и отдельных его участников.
Несмотря на то, что некоторые материалы по сей день остаются засекреченными, к настоящему времени вполне ясны политические и иные причины, приведшие к началу активных исследований в области атомной физики в середине XX века и являвшиеся затем их катализатором.
Достаточно простой и понятной кажется теперь даже сама физика процессов, протекающих в атомной и водородной бомбах.

На сегодняшний день лишь немногие страницы Манхэттенского проекта остались неосвещенными должным образом. 
Одна из таких страниц - использовавшаяся при разработке ядерного оружия математика.
Математический аппарат и ученые-математики сыграли исключительно важную роль в Манхэттенском проекте.
Именно тогда, в условиях крайней опасности и даже невозможности проведения большого числа физических экспериментов математические модели продемонстрировали свою исключительную эффективность.
Тогда же стала понятна огромная важность применения компьютеров для быстрого проведения численных симуляций физических процессов и быстрой проверки различных физических гипотез.

Данная книга не касается морально - нравственных, экономико - политических и иных вопросов гуманитарного характера, непременно связанных с созданием и использованием ядерного оружия.
Вместо этого в книге предпринята попытка собрать воедино основные математические методы и идеи, так или иначе повлиявшие на его создание, и получившие широкое применение впоследствии.
Главной целью было наиболее простым и понятным образом описать математический аппарат, использовавшийся при разработке ядерного оружия. 
Главной причиной, почему это оказалось возможным, является его выразительность и относительная простота. 
Многие считают эту неожиданную простоту \textit{математики Манхэттенского проекта} одним из главных факторов его впечатляющего успеха.

Эта книга, изначально задумывавшаяся как перечень матметодов Манхэттенского проекта в их аутентичной форме, претерпела значительные изменения в процессе своего написания и в итоге включает в себя гораздо больше, чем задумывалось изначально.
Акцент книги сместился с изложения сугубо математических задач и методов их решения на то, как эти задачи вписывались в общую картину создания ядерного оружия.

Рассуждения, выкладки и формулировки некоторых теорем часто приводятся в упрощенном виде и не всегда соответствуют стандартам строгости, принятым в современной математике.
Описываемые методы и подходы не всегда использовались в годы Манхэттенского проекта именно в том виде, в каком приводятся здесь.
Наиболее сложные места в книге выделены в главы, которые можно пропустить без ущерба для общей картины.

\\

[....благодарности....]
\\

Эта книга явилась плодом многолетнего увлечения автора вопросами истории, теории и математических моделей ядерной физики начала-середины XX века. 
Искренне надеюсь, что изложение покажется простым и понятным даже читателям, не обладающим серьезной математической подготовкой, а чтение книги доставит такое же удовольствие, какое испытывал автор при ее написании.

