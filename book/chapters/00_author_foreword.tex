\chapter*{Предисловие автора}

6 и 9 августа 1945 года на Хиросиму и Нагасаки были сброшены атомные бомбы общей мощностью около 35 килотонн в тротиловом эквиваленте.
Взрывы произошли в 500 метрах над землей, сформировав сильнейшую ударную волну и выделив огромное количество тепла и радиации.
Более 180000 человек погибло в момент взрыва и в последующие дни от ранений и лучевой болезни.  

Фактически, в этом уже не было необходимости. 
Вторая Мировая война подходила к своему завершению. 
Германия капитулировала, Япония была как никогда близка к капитуляции.
Атомные бомбардировки в основном были призваны продемонстрировать всему миру беспрецедентную военную мощь США. 
Мощь, явившуюся результатом, пожалуй, самого амбициозного проекта в истории науки, объединившего сотни лучших ученых своего времени - проекта ``Манхэттен''. 

Манхэттенский проект подвел черту под почти полувековыми размышлениями ученых о строении атома. 
Он получил статус национального и по сей день является по сути единственным примером столь успешного объединения лучших умов своего времени для достижения вполне практической и понятной каждому цели.

Применение атомного оружия не оставило равнодушным никого.
Весь мир от простых людей до ученых и политиков активно обсуждал нравственные и политические вопросы использования атомной энергии.
Кто-то искренне поражался силе научной мысли, освободившей энергию атома.
Другие решительно критиковали столь безответственное, по их мнению, применение полученных знаний о природе.
Руководители стран Европы, Советского Союза и Востока понимали, что США впервые в истории удалось получить в свои руки оружие беспрецедентной разрушительной силы. 
Это был именно тот эффект, на который надеялись основные заказчики атомного проекта в США - политики и военные.  
Началась ядерная гонка, продлившаяся почти полвека и не полностью оконченная и по сей день.

Отношение самих ученых к успеху атомного проекта было неоднозначным.
Многие из тех, кто призывал к созданию атомного оружия для победы над фашистской Германией и в конце концов осознал, что Германия не обладает ничем подобным, стали активно пропагандировать не использовать его и даже прекратить исследования в этой области. 
Сам Альберт Эйнштейн, подписавший в 1939 году знаменитое "письмо Эйнштейна Рузвельту" с призывом к созданию атомного оружия, позднее отчаянно критиковал его разработку и применение в военных целях.

Другие же ученые, напротив, продолжили активно работать в направлении увеличении эффективности и мощи ядерного оружия.
Кто-то не видел своей личной вины за произошедшее, другие подозревали Советский Союз в активизации работ к этом направлении.
Иные понимали, что работать бок о бок с лучшими умами своего времени и над столь масштабными проектами им, возможно, уже не придется. 
Так или иначе, после событий августа 1945 года исследовательская лаборатория в Лос-Аламосе, в которой родился проект ``Манхэттен'', после небольшой передышки начала работу над еще более смертоносным оружием - водородной бомбой.
Далеко не все подразделения трудились в прежнем составе, но и того, что осталось, вполне хватило для успешного завершения и этого проекта.  

Интересен и другой эффект событий августа 1945 года, изменивших отношение к науке и ученым в глазах простых людей.
Именно благодаря беспрецедентной эффективности и разрушительной мощи ядерного оружия репутация ученых среди простых людей взлетела до небес. 
До Манхэттенского проекта на ученых смотрели как на людей чудаковатых, но безобидных и погрязших в своих никому больше непонятных задачах. 
Образ эксцентричного ученого имел свой шарм, то и дело всплывая то в фантастической литературе, то в кино.
После Второй Мировой войны отношение к ученым изменилось кардинально.
Весь мир узнал, насколько разрушительными могут быть идеи, рождающиеся на кончике пера и университетских досках.
Человечество, на протяжении всей своей истории принимая дары науки в лучшем случае с неосознанной благодарностью, в большинстве своем просто не было готово к столь кардинальным и опасным изменениям в их жизни.
У многих возник естесственный страх перед той разрушительной мощью, которую наука без особого труда могла призвать в этот мир. 

Многое сказано и написано о Манхеттенском проекте.
Вполне понятны политические и иные причины, приведшие к началу активных исследований в области атомной физики.
Достаточно простой и понятной кажется теперь и сама физика процессов, протекающих в атомной бомбе.
Немногие страницы Манхэттенского проекта остались должным образом неосвещенными. 
Одна из таких страниц - математический аппарат и ученые-математики, сыгравшие исключительно важную роль в проекте.

В данной книге мы не рассматриваем морально-нравственные, экономико-политические и иные вопросов гуманитарного характера, неразрывно связанные с созданием и использованием атомной и водородной бомб.
Вместо этого мы сконцентрируемся на чисто научной-теоретической стороне процесса создания атомной и водородной бомб и, в особенности использованного в этой области математического аппарата.
Несомненным плюсом здесь является его крайняя выразительность и относительная простота, которую многие считают одним из главных факторов успеха Манхэттенского проекта.
Изложение материала в книге призвано . Приводимые рассуждения, выкладки, утверждения и теоремы иногда не соответствовуют стандартам строгости, принятым в современной математике. 

просто, не всегда строго .............. не всегда именно то, что было на тот момент.


[....благодарности....]

Эта книга явилась плодом многолетнего увлечения автора вопросами теории, применяемой в атомной физике. 
Искренне надеюсь, что изложение покажется простым и понятным даже читателям, не обладающим серьезной математической подготовкой, и доставит такое же удовольствие, какое испытывал я при ее написании.

------------------------ IDEAS ------------------------ 
















