\chapter*{Предисловие автора}
\addcontentsline{toc}{chapter}{Предисловие автора}

6 и 9 августа 1945 года на Хиросиму и Нагасаки были сброшены атомные бомбы общей мощностью около 35 килотонн в тротиловом эквиваленте.
Взрывы произошли в 500 метрах над землей, сформировав сильнейшую ударную волну и выделив огромное количество тепла и радиации.
Более 180000 человек погибло в момент взрыва и в последующие дни от ранений и лучевой болезни.
15 августа Япония объявила о своей капитуляции, 2 сентября был подписан формальный акт.

Фактически, в атомных бомбардировках уже не было необходимости. 
Вторая Мировая война подходила к своему завершению. 
Германия капитулировала, Япония была как никогда близка к этому.
Атомные бомбардировки в основном были призваны продемонстрировать всему миру беспрецедентную военную мощь США.
Мощь, явившуюся результатом, пожалуй, самого амбициозного проекта в истории науки, объединившего сотни лучших ученых своего времени - проекта ``Манхэттен''.

Применение атомного оружия не оставило равнодушным никого.
Весь мир начиная с простых людей до ученых и политиков активно обсуждал нравственные и политические вопросы использования атомной энергии.
Кто-то искренне поражался силе научной мысли, освободившей энергию атома.
Другие решительно критиковали столь безответственное, по их мнению, применение полученных знаний о природе.
Руководители стран Европы, Советского Союза и Востока понимали, что США впервые в истории удалось получить в свои руки оружие беспрецедентной разрушительной силы. 
Это был именно тот эффект, на который надеялись основные заказчики атомного проекта в США - политики и военные.  
Началась ядерная гонка, продлившаяся несколько десятилетий и не полностью оконченная по сей день.

Отношение самих ученых к успеху атомного проекта США было неоднозначным.
Многие из тех, кто призывал к созданию атомного оружия для победы над фашистской Германией и в конце концов осознал, что Германия не обладает ничем подобным, стали активно пропагандировать не использовать его и даже прекратить исследования в этой области. 
Сам Альберт Эйнштейн, подписавший в 1939 году знаменитое ``письмо Эйнштейна Рузвельту'' с призывом к созданию атомного оружия, позднее отчаянно критиковал его разработку и применение в военных целях.

Другие же ученые, напротив, продолжили активно работать в направлении увеличении эффективности и мощи ядерного оружия.
Кто-то не видел своей личной вины за произошедшее в Хиросиме и Нагасаки, другие подозревали Советский Союз в активизации работ к этом направлении.
Иные понимали, что работать бок о бок с лучшими умами своего времени и над столь масштабными проектами им, возможно, уже не придется. 
Так или иначе, после событий августа 1945 года исследовательская лаборатория в Лос-Аламосе, в которой всего два года тому назад родился проект ``Манхэттен'', после небольшой передышки начала работу над еще более смертоносным оружием - водородной бомбой.
Далеко не все подразделения трудились в прежнем составе, но и того, что осталось, вполне хватило для успешного завершения и этого проекта. 
Полномасштабные испытания 1 ноября 1952 года явили миру мощь термоядерного оружия, гораздо более разрушительного, нежели ядерное.
Испытательный взрыв, осуществленный США на атолле Эниветок, по меньшей мере в 1000 раз превосходил по мощности бомбу, сброшенную на Хиросиму.

Годом позже, в 1953-м, к термоядерной гонке присоединился и СССР, взорвав сначала более слабый заряд, чем у США, но быстро нарастив темпы. 
Затем к гонке присоединились Великобритания, Китай и Франция.
Мысли о повороте разрушительной мощи атома в мирное русло и разоружении возникли у политиков только во время Карибского кризиса в 1962 году, когда мир стоял на грани ядерной катастрофы. 
И лишь еще через десять лет, в 1972-м году, между СССР и США был подписан первый двусторонний договор, запрещающий сторонам наращивать ядерный арсенал.

Интересен и другой, более мирный эффект событий 40-х - 50-х годов XX века, изменивший отношение к науке и ученым.
Именно благодаря невиданной ранее мощи ядерного оружия репутация ученых среди простых людей взлетела до небес. 
До Второй Мировой войны на ученых смотрели как на людей чудаковатых, но безобидных и погрязших в своих никому больше непонятных задачах. 
Образ эксцентричного ученого имел свой шарм, то и дело всплывая в фантастической литературе и кино.
После Второй Мировой войны отношение к ученым сильно изменилось.
Весь мир узнал, насколько разрушительными могут быть идеи, рождающиеся на кончике пера и университетских досках.
Человечество, на протяжении всей своей истории принимая дары науки в лучшем случае с неосознанной благодарностью, в большинстве своем просто не было готово к столь кардинальным и потенциально опасным изменениям в их жизни.
У многих возник естесственный страх перед той разрушительной мощью, которую наука без особого труда могла призвать в этот мир. 

О самом Манхэттенском проекте сказано и написано довольно много.
С лица проекта постепенно исчезли темные пятна с гифами секретности, стала понятна мотивация заинтересованных сторон и отдельных его участников.
Несмотря на то, что некоторые материалы по сей день остаются за секреченными, к настоящему времени вполне ясны политические и иные причины, приведшие к началу активных исследований в области атомной физики в середине XX века и явившиеся затем их двигателем.
Достаточно простой и понятной кажется теперь и сама физика процессов, протекающих в атомной и водородной бомбах.

На сегодняшний день лишь немногие страницы Манхэттенского проекта остались неосвещенными должным образом. 
Одна из таких страниц - математический аппарат и ученые-математики, сыгравшие исключительно важную роль в проекте.
Именно в Манхэттенском проекте, в условиях крайней опасности и даже невозможности проведения большого числа физических экспериментов математические модели продемонстрировали свою исключительную эффективность.
Именно тогда стала понятна огромная важность применения вычислительных машин для быстрого проведения численных симуляций физических процессов и быстрой проверки многочисленных физических гипотез.

В данной книге мы не касаемся морально-нравственных, экономико-политических и иных вопросов гуманитарного характера, непременно связанных с созданием и использованием атомной и водородной бомб.
Вместо этого в книге предпринята попытка собрать воедино основные математические методы и идеи, так или иначе повлиявшие на создание атомного оружия, и получившие широкое применение впоследствии.
Несомненным плюсом здесь является крайняя выразительность и относительная простота использовавшегося при разработке ядерного оружия математического аппарата. 
Эту неожиданную простоту многие считают одним из главных факторов успеха Манхэттенского проекта.

Изложение материала в книге призвано представить основные математические идеи, использованные при разработке ядерного оружия, наиболее простым и понятным образом. 
Наиболее сложные места выделены в главы, которые можно пропустить без ущерба для общей картины.
Приводимые в книге математические рассуждения, выкладки и формулировки некоторых утверждений и теорем не всегда соответствуют стандартам строгости, принятым в современной математике.
Описываемые методы и подходы не всегда использовались в годы Манхэттенского проекта именно в том виде, в каком приводятся здесь.

[....благодарности....]

Эта книга явилась плодом многолетнего увлечения автора вопросами теории, применяемой в атомной физике. 
Искренне надеюсь, что изложение покажется простым и понятным даже читателям, не обладающим серьезной математической подготовкой, а чтение книги доставит такое же удовольствие, какое испытывал автор при ее написании.


------------------------ IDEAS ------------------------ 











