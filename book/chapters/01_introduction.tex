\chapter*{Введение}

XX век был поистине богатым научными открытиями в самых разных областях науки. Ученые как никогда приблизились к пониманию механики как микро, так и макро процессов окружающего мира. В биологии был обнаружен и описан основной строительный блок всего живого - молекула ДНК. Стремительно начала развиваться генная инженерия, находя приложения в самых разных отраслях человеческой деятельности. В физике были открыты общая и специальная теории относительности, квантовая механика. Выдающиеся достижения физиков и биологов активно освещались в прессе и практически сразу становились предметом жаркого обсуждения даже людьми, далекими от мира науки и в лучшем случае довольно приблизительно понимающими, о чем идет речь. 
Подобного, к сожалению, нельзя сказать об отношении к достижениям математики XX века - кроме самих математиков и, пожалуй, некоторых физиков, о них не знал практически никто. А они были поистине впечатляющими, вполне сравнимыми по потенциальной мощи с квантовой механикой или открытием ДНК. Стоит упомянуть хотя бы появление и активное использование компьютеров, необходимых для сложных расчетов тогда и распространенных повсеместно сейчас.

Отсутствие должного освещения открытий математики отчасти связано с самой спецификой данной науки. Лишь в редких случаях по-настоящему сложную математическую теорию можно объяснить широкому кругу людей-непрофессионалов. Чувство красоты математических рассуждений, доказательств и окончательных выводов необходимо упорно воспитывать в себе некоторое время, прежде чем появится понимание того, что стоит за длинными формулами и придет осознание того, как полученные выводы можно применить на практике.
Данная книга призвана восполнить этот пробел и рассказать, какую роль сыграли математики в знаменитом манхэттенском проекте, явившим миру всю мощь ядерной энергии. Я попытаюсь осветить мат. аппарат, который использовался при расчетах, связанных с конструированием атомной и водородных бомб, уделяя особое внимание методам, созданным именно в процессе работы над проектом “Манхэттен”.

Книга рассчитана на широкий круг читателей, интересующихся математикой и ее приложениями в ядерной физике. Книга будет интересна студентам и аспирантам физико-математических специальностей, а также просто интересующиеся тематикой атомной физики начала-середины XX века и применяемого там мат. аппарата. От читателей в большинстве случаем требуются лишь общие знания об основных понятиях математики - множествах и отображениях. Для понимания наиболее сложных моментов книги будет полезна специальная подготовка в рамках не ниже 2 курса физико-математических специальностей, общие знания по математическому анализу, теории вероятностей, дифференциальным уравнениям и функциональному анализу.
