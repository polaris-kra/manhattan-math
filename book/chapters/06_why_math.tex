\chapter{Зачем нужна математика?}\label{ch:why_math}

\epigraph{\emph{Каждая включенная в книгу формула \break уменьшает число ее читателей вдвое.}}{Стивен Хокинг, Роджер Пенроуз}

TODO

------------------------ IDEAS ------------------------ 

На этом подготовительную часть книги можно считать законченной.
Мы знаем, какие физические процессы стоят за созданием атомного оружия ([стоит ли упирать на оружие]) и в целом понимаем суть связанных с этим задач.
Мы представляем, какая работа была у физиков....
Что же делали математики? ....


Прежде чем пытаться разобраться в конкретных уравнениях и моделях, полезно значть в целом, а что же вообще способна дать математика?


Дмитрий Иванович Менделеев утверждал, что наука начинается там, где начинают считать

Математика - это организованные рассуждения


Попытки разобраться в сути какого-либо уже исследованном кем-то ранее явлении реального мира чем-то напоминают процесс очистки гипотетического фрукта с многослойной кожурой.
Первым и самым простым слоем являются личный опыт, мнения других людей и ``авторитетных'' источников о данном вопросе. 
На этом, собственно, можно и остановиться, сказав, что достаточно разобрались в вопросе.

Если полученные ответы нас не устраивают, не понятны, либо не полны и желание разобраться в сути явления не угасло, то придется перейти к следующему слою - предметной области явления, например, физике.
Необходимо хотя бы в общих чертах понять, что же именно происходит в интересующем нас явлении природы. 
Какие объекты в нем участвуют и по каким правилам взаимодействуют друг с другом. Какие моменты существенны, а какими можно пренебречь.
Продвинувшись в понимании физической сути процесса, мы 

Наконец, последний и традиционно самый трудный слой - математика явления.
Каждое явление природы имеет свой язык описания  ...  сложно .. вместо объектов - абстракции, вместо простых правил взаимодействия - сложные уравнения.

---

физика - ранее умение делать открытия зависело от того, насколько наблюдателен был ученый, насколько хорошо он умел проводить параллели между уже известными ялениями и только изучаемыми.
Движения огромных небесных тел описывалось исходя из аналогичных движений, которые можно было повторять в удобном масштабе в своей алборатории и т.п. [еще примеры]
Новая физика потребовала от ученых вообразить нечто не имевшее аналогов с ранее изученным в принципе. 
Это восхищало даже далеких от физики современников.
В математике такие штуки привыкли проворачивать довольно давно. 
Стефан Банах, один из творцов математики в ее современном виде, говорил: "хорошие математики видят аналогии, лучшие могут видеть аналогии между аналогиями". Сам он, безусловно, был одним из лучших.
