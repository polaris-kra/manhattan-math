\begin{thebibliography}{9}

\bibitem{bib_mathphylhist} 
Радул Д.Н. 
\textit{История и философия науки: философия математики}. 2-е изд., испр. и доп. Учебное пособие для вузов. 2019.

\bibitem{bib_archimed} 
Архимед.
\textit{Сочинения}. М., 1962.

\bibitem{bib_feynmanphyslaws} 
Фейнман Р.
\textit{Характер физических законов}. М.: Издательство АСТ, 2018.

\bibitem{bib_lomonosov} 
Ломоносов М.В.
\textit{Полное собрание сочинений}. Изд. АН СССР, М.-Л., 1951

\bibitem{bib_lomonosov} 
Арнольд В.И.
\textit{Истории давние и недавние}. М., ФАЗИС, 2002.

\bibitem{bib_chemhist} 
Левченков С.И.
\textit{Краткий очерк истории химии}. Ростов н/Д: Изд-во Рост. ун-та, 2006.

\bibitem{bib_antique_philosophy} 
Шишкоедов П.
\textit{Философия античности}. Издательские решения, 2015.

\bibitem{bib_democrit} 
Лурье С.Я.
\textit{Демокрит}. Л.: Наука, 1970.

\bibitem{bib_ancient_atomism} 
Лурье С.Я.
\textit{Теория бесконечно малых у древних атомистов}.  М.-Л.: АН СССР, 1935.

\bibitem{bib_chemistry} 
Третьяков Ю.Д., Олейников Н.Н., Кеслер Я.А., Казимирчик И.В.
\textit{Химия: справочные материалы}. М.: Просвещение, 1989.

\bibitem{bib_antique_philosophy} 
Шамшин Д.Л.
\textit{Химия: Учеб. пособие}. М.: Высшая школа, 1980.

\bibitem{bib_niels_bohr_1} 
Niels Bohr
\textit{The Philosophical Writings of Niels Bohr}. Ox Bow Press, 1987


\end{thebibliography}
